% !TeX program = xelatex
% ----------------------------------------------------------------------
% 1. 文档类: 直接使用ctexart, 并移除冗余的UTF8选项
% ----------------------------------------------------------------------
\documentclass[10pt,twoside]{ctexart}

% ----------------------------------------------------------------------
% 2. 宏包加载
% ----------------------------------------------------------------------
% 几何尺寸与页眉页脚
\usepackage[top=2.5cm,bottom=2.5cm,left=2cm,right=2cm]{geometry}
\usepackage{fancyhdr}

% 图形与排版
\usepackage{graphicx}
\usepackage{multicol}
\usepackage{booktabs}
\usepackage{array}
\usepackage{float}
\usepackage{listings}
\usepackage{xcolor}

% 数学公式
\usepackage{amsmath,amssymb}

% 图表标题
\usepackage[font=small,labelfont=bf]{caption}
\usepackage{subcaption}

% 参考文献引用
\usepackage{cite}

% 超链接 (建议放在后面加载)
\usepackage[hidelinks]{hyperref}

% ----------------------------------------------------------------------
% 3. 字体设置 (使用ctex的现代命令)
% ----------------------------------------------------------------------
\setCJKmainfont[BoldFont=Heiti SC, ItalicFont=Kaiti SC]{Songti SC}
\setCJKsansfont{PingFang SC}
\setCJKmonofont{STFangsong}

% ----------------------------------------------------------------------
% 4. 自定义样式与命令
% ----------------------------------------------------------------------
% 代码环境
\lstset{
    basicstyle=\small\ttfamily,
    breaklines=true,
    commentstyle=\color{gray},
    keywordstyle=\color{blue},
    language=Python,
    numbers=left,
    numberstyle=\tiny\color{gray},
    showstringspaces=false,
    stringstyle=\color{red},
    captionpos=b % 将代码标题置于底部
}

% 页眉页脚
\pagestyle{fancy}
\fancyhf{}
\fancyhead[CE]{\small 北~京~交~通~大~学}
\fancyhead[CO]{\small BEIJING JIAOTONG UNIVERSITY}
\fancyfoot[C]{\thepage}
\renewcommand{\headrulewidth}{0.4pt}
\renewcommand{\footrulewidth}{0pt}
\fancypagestyle{plain}{\pagestyle{fancy}}


% 章节标题格式
\makeatletter
\renewcommand\section{\@startsection{section}{1}{\z@}%
  {3.5ex \@plus 1ex \@minus .2ex}% 
  {2.3ex \@plus.2ex}%
  {\normalfont\large\bfseries}}
\renewcommand\subsection{\@startsection{subsection}{2}{\z@}%
  {3.25ex\@plus 1ex \@minus .2ex}%
  {1.5ex \@plus .2ex}%
  {\normalfont\normalsize\bfseries}}
\makeatother

% 图表标题格式
\captionsetup{font=small,labelfont=bf}

% 摘要与关键词环境
\renewenvironment{abstract}{%
  \noindent\textbf{摘\quad 要:}}{\par}
\newcommand{\keywords}[1]{%
  \noindent\textbf{关键词:}#1\par}

% 自定义英文摘要环境
\newenvironment{enabstract}{%
  \noindent\textbf{Abstract: }}{\par}
\newcommand{\enkeywords}[1]{%
  \noindent\textbf{Keywords: }#1\par}

% ==================== 序言结束 ====================

% 标题与作者信息
\title{%
    {\LARGE\bfseries 在此处填写你的实验报告标题}\\[0.5cm]
    {\large 课程名称:在此处填写课程名称}%
}
\author{%
    {\large 姓名:苏长皓 \qquad 学号:23331147}\\[0.2cm]
    {\small (北京交通大学~詹天佑学院,北京)}%
}
\date{{\large (实验日期:\today)}}


\begin{document}

% 标题部分
\maketitle
\thispagestyle{fancy}

% 中文摘要
\begin{abstract}
[ 在此处填写中文摘要内容。简要介绍实验的背景、目的、方法、主要结果和结论。]
\end{abstract}

\keywords{关键词1;关键词2;关键词3}

\vspace{0.5cm}

% 英文标题和摘要
\begin{center}
{\large\bfseries English Title of Your Experiment Report}\\[0.3cm]
{\normalsize NAME Pinyin}\\[0.2cm]
{\small (School of Electronic and Information Engineering, Beijing Jiaotong University, Beijing 100044, China)}\\[0.2cm]
{\small (Date: \today)}
\end{center}

\begin{enabstract}
[ Type your English abstract here. It should be a concise summary of the experiment's background, objectives, methods, key findings, and conclusions. ]
\end{enabstract}

\enkeywords{Keyword1; Keyword2; Keyword3}

\vspace{0.5cm}

% 正文开始,使用双栏
\begin{multicols}{2}
\raggedcolumns

\section{实验目的}
1. [ 目的1 ]
2. [ 目的2 ]
3. [ 目的3 ]

\section{实验原理}
[ 在此处详细阐述实验所依据的理论基础、核心公式和工作原理。]

\subsection{原理一}
[ 对第一个核心原理进行说明。]

这是一个公式示例:
\begin{equation}
    E = mc^2
    \label{eq:einstein}
\end{equation}
如公式 \ref{eq:einstein} 所示...

\subsection{原理二}
[ 对第二个核心原理进行说明。]

\section{实验内容与步骤}
[ 在此处描述具体的实验内容、所使用的仪器设备以及详细的操作步骤。]

\subsection{实验设备}
\begin{itemize}
    \item 设备一 (型号)
    \item 设备二 (型号)
\end{itemize}

\subsection{实验步骤}
1. 步骤一...
2. 步骤二...
3. 步骤三...

\section{实验结果与分析}
[ 在此处展示实验数据、图表、波形等,并对结果进行深入的分析和讨论。]

\begin{figure}[H]
    \centering
    % 请将图片文件放在与.tex文件相同的目录下,或指定正确路径
    % \includegraphics[width=\columnwidth]{your_image_file.png}
    \caption{这是一个图片标题示例。}
    \label{fig:example}
\end{figure}

\begin{table}[H]
    \centering
    \caption{这是一个表格标题示例。}
    \label{tab:example}
    \begin{tabular}{ccc}
    \toprule
    参数 A & 参数 B & 结果 C \\
    \midrule
    数据1 & 数据2 & 数据3 \\
    数据4 & 数据5 & 数据6 \\
    \bottomrule
    \end{tabular}
\end{table}

\section{结论与思考}
[ 在此处总结实验结论,并提出实验过程中遇到的问题、改进建议或进一步的思考。这是一个引用示例 \cite{placeholder}。]

\end{multicols}

% 参考文献部分
\vspace{1cm}
\noindent\rule{\textwidth}{0.4pt}
\begin{thebibliography}{99}
\footnotesize
% 请在此处添加你的参考文献

\end{thebibliography}

\end{document}